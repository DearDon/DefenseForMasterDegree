\documentclass[compress]{beamer}
%为了打印[]中还可加入handout或tran
%compress用于压缩侧边栏和上下两端的导航条,使文档紧凑
\usepackage{ctex}
%为支持汉字,由于ctex文档类中没有beamer的,故用普通文档类,再引入ctex包即可
%支持中文,其实该包会默认在document环境中加入CJK环境以支持中文,所以cjk包的
%命令它都支持
\usepackage{pgf,pgfarrows,pgfnodes,pgfautomata,pgfheaps}
%上面的包都是一些用pgf分布图的包
\usepackage{graphicx}%用于插入图片
\usepackage{beamerthemesplit}
\usepackage{multicol}
\usepackage{multirow}

%下面用于自定义模板

\beamertemplateshadingbackground{red!10}{structure!10}
%设置背景色由%10的红变为%10的结构颜色

\beamertemplatetransparentcoveredhigh
%隐藏文本高度透明

\beamertemplatetransparentcovereddynamicmedium
%使所有被隐藏的文本完全透明,动态

%\usetheme{Warsaw}
\usetheme{Singapore}
%\usetheme{Berkeley}
%\usetheme{Berlin}
%使用主题

\setbeamertemplate{footline}[page number]{}
%除掉页面下方的信息条
\setbeamertemplate{navigation symbols}{}
%除掉页面下方导航条

\begin{document}
\title{波形拟合反演震源机制的定权研究及误差评定}
\author{邓东平 2013202140004\\
	导师:朱良保教授}
\institute{\includegraphics[height=3cm]{figures/whulogo.eps}\\
	武汉大学测绘学院}
\date{}
\AtBeginSection[]{
	\frame<handout:0>{
		\frametitle{预览}
		\tableofcontents[current,currentsubsection]
	}
}


\frame{\titlepage}
\transdissolve<5>
\section*{}
\frame{
\frametitle{概览}
\tableofcontents
}
\section{研究意义}
\frame{
	\frametitle{研究意义}
	\begin{itemize}
		\item 发震构造研究、灾害评估
		\item 区域应力、地震活动性
		\item 介质结构、海啸模拟等研究
	\end{itemize}
}
\section{研究现状和本文目标}
\subsection{研究现状}
\frame{
	\frametitle{研究现状}
	\begin{itemize}
		\item<+->{原理:}
			\begin{equation}
			\label{eq01}
			\left\{
			 \begin{array}{l}
			    U_z(r,\phi,0,\omega)=Z_{SS}{\cdot}s_2+Z_{DS}{\cdot}s_3+Z_{DD}{\cdot}s_1\\
			    U_r(r,\phi,0,\omega)=R_{SS}{\cdot}s_2+R_{DS}{\cdot}s_3+R_{DD}{\cdot}s_1\\
				U_{\phi}(r,\phi,0,\omega)=T_{SS}{\cdot}t_2+T_{DS}{\cdot}t_1\\
			 \end{array}
			\right.
			\end{equation}
		\item<+->{方法:}波形拟合(波形数据),格点搜索(公式\ref{eq01}非线性),
		\item<+->{应用:}CAP,CPS等代表性方法(程序)广泛应用
	\end{itemize}
}
\frame{
	\frametitle{研究现状}
	\begin{itemize}
		\item<+->{优点:}
			\begin{enumerate}[<+->]
			\item 波形数据充分应用了地震波信息
			\item 震源机制解空间较小,且正演合成迅速,格点搜索可快速反演
			\end{enumerate}
		\item<+->{问题:}
			\begin{enumerate}[<+->]
			\item 无法直接给出误差评价,无法有效识别病态问题
			\item CAP和CPS的加权方案不一致,数值相对大小冲突
			\end{enumerate}
	\end{itemize}
}
\subsection{本文目标}
\frame{
	\frametitle{\subsecname}
	\begin{itemize}
		\item 统一优化定权
		\item 针对CAP、CPS给出结果误差评价
	\end{itemize}
}

\section{解决方案}
\frame{
\frametitle{\subsecname}
	\begin{itemize}
		\item 优化定权
			\begin{enumerate}
			\item 分析二者定权的理论依据,联合定权解决差异
			\item 数值定量精化,结果尽量客观
			\end{enumerate}
		\item 针对CAP、CPS给出结果误差评价
			\begin{enumerate}
			\item 估计数据噪声
			\item 计算震源机制协方差矩阵
			\end{enumerate}
	\end{itemize}
}
\subsection{优化加权}
\frame{
\frametitle{优化加权}
	\begin{itemize}
		\item<+-> 联合加权
			\begin{enumerate}[<+->]
			\item CPS加权$W1$,考虑信噪比,权重随震中距单调递减
			\item CAP加权$W2$,考虑振幅调节,权重随震中距单调递增
			\item 信噪比和振幅调节均可提高数据质量,应联合统一$WT=W1*W2$
			\end{enumerate}
		\item<+-> 定量精化
			\begin{enumerate}[<+->]
			\item 利用震中距估计的$W1,W2$均较粗糙,改为由波形数据直接定量计算
			\item CAP加权估计$W2$的公式$(r/r_0)^p$中的参考参数$r_0,p$只能经验判定,主观性很强
			\end{enumerate}
		\item<+-> 本文最终权重
			\begin{enumerate}
			\item $WT=(1-NoiseStd/WaveStd)/L2norm$
			\end{enumerate}
	\end{itemize}
}
\subsection{误差估计}
\frame{
\frametitle{误差估计}
	\begin{columns}
	\column{0.4\textwidth}<1->
	\includegraphics[height=7.5cm]{figures/flowchart.pdf}
	\column{0.6\textwidth}<1->
	\begin{itemize}
		\item<+-> STEP1 估计数据噪声
			\begin{enumerate}[<+->]
			\item 截取震前平静期数据样本
			\item 参数估计得到噪声分布函数(高斯)
			\end{enumerate}
		\item<+-> STEP2 随机生成模拟数据集
			\begin{enumerate}[<+->]
			\item 用噪声分布函数随机生成噪声数据
			\item 噪声数据与原始观测数据叠加,生成多组模拟观测数据
			\end{enumerate}
		\item<+-> STEP3 反演得解集并计算协方差矩阵
			\begin{enumerate}[<+->]
			\item 每组"观测"数据独立反演,得随机误差范围内解集
			\item 对解集统计分析,计算震源机制三参数协方差矩阵
			\end{enumerate}
	\end{itemize}
	\end{columns}
}
\section{实践检验}
\subsection{理论实验}
\frame{
	\frametitle{\subsecname}
	\begin{itemize}
		\item<+-> Install
		\item<+-> learn
		\item<+-> practise
	\end{itemize}
}
\subsection{实例应用}
\frame{
	\frametitle{\subsecname}
	\begin{itemize}
		\item<+-> Install
		\item<+-> learn
		\item<+-> practise
	\end{itemize}
}
\section{总结和展望}
\frame{
	\frametitle{secname}
	\begin{itemize}
		\item<+-> Install
		\item<+-> learn
		\item<+-> practise
	\end{itemize}
}
\end{document}
