\documentclass[compress]{beamer}
%为了打印[]中还可加入handout或tran
%compress用于压缩侧边栏和上下两端的导航条,使文档紧凑
\usepackage{ctex}
%为支持汉字,由于ctex文档类中没有beamer的,故用普通文档类,再引入ctex包即可
%支持中文,其实该包会默认在document环境中加入CJK环境以支持中文,所以cjk包的
%命令它都支持
\usepackage{pgf,pgfarrows,pgfnodes,pgfautomata,pgfheaps}
%上面的包都是一些用pgf分布图的包
\usepackage{graphicx}%用于插入图片
\usepackage{beamerthemesplit}
\usepackage{multimedia}

%下面用于自定义模板

\beamertemplateshadingbackground{red!10}{structure!10}
%设置背景色由%10的红变为%10的结构颜色

\beamertemplatetransparentcoveredhigh
%隐藏文本高度透明

\beamertemplatetransparentcovereddynamicmedium
%使所有被隐藏的文本完全透明,动态

%\usetheme{Warsaw}
\usetheme{Singapore}
%\usetheme{Berkeley}
%\usetheme{Berlin}
%使用主题

\setbeamertemplate{footline}[page number]{}
%除掉页面下方的信息条
\setbeamertemplate{navigation symbols}{}
%除掉页面下方导航条

\begin{document}
\title{波形拟合反演震源机制的定权研究及误差评定}
\author{邓东平 2013202140004\\
	导师:朱良保教授}
\institute{\includegraphics[height=3cm]{figures/whulogo.eps}\\
	武汉大学测绘学院}
\date{}
\AtBeginSection[]{
	\frame<handout:0>{
		\frametitle{进度}
		\tableofcontents[current,currentsubsection]
	}
}


\frame{\titlepage}
\transdissolve<5>
\section*{导航}
\frame{
\frametitle{概览}
\tableofcontents
}
\section{研究意义和现状}
\frame{
	\frametitle{研究意义}
	\begin{itemize}
		\item 发震构造研究、灾害评估
		\item 区域应力、地震活动性
		\item 介质结构、海啸模拟等研究
	\end{itemize}
}
\frame{
	\frametitle{研究现状}
	\begin{itemize}
		\item<+->{原理:}
			\begin{equation}
			\label{eq01}
			\left\{
			 \begin{array}{l}
			    U_z(r,\phi,0,\omega)=Z_{SS}{\cdot}s_2+Z_{DS}{\cdot}s_3+Z_{DD}{\cdot}s_1\\
			    U_r(r,\phi,0,\omega)=R_{SS}{\cdot}s_2+R_{DS}{\cdot}s_3+R_{DD}{\cdot}s_1\\
				U_{\phi}(r,\phi,0,\omega)=T_{SS}{\cdot}t_2+T_{DS}{\cdot}t_1\\
			 \end{array}
			\right.
			\end{equation}
		\item<+->{方法:}波形拟合(波形数据),格点搜索(公式\ref{eq01}非线性),
		\item<+->{应用:}CAP,CPS等代表性方法(程序)广泛应用
	\end{itemize}
}
\frame{
	\frametitle{研究现状}
	\begin{itemize}
		\item<+->{优点:}
			\begin{enumerate}[<+->]
			\item 波形数据充分应用了地震波信息
			\item 震源机制解空间较小,且正演合成迅速,格点搜索可快速反演
			\end{enumerate}
		\item<+->{问题:}
			\begin{enumerate}[<+->]
			\item 无法直接给出误差评价,无法有效识别病态问题
			\item CAP和CPS的加权方案不一致,数值相对大小冲突
			\end{enumerate}
	\end{itemize}
}

\section{研究目标和解决方案}
\subsection{研究目标}
\frame{
	\frametitle{研究目标}
	\begin{itemize}
		\item<+-> 统一优化定权
		\item<+-> 针对CAP、CPS给出结果误差评价
	\end{itemize}
}
\subsection{解决方案}
\frame{
\frametitle{\subsecname}
	\begin{itemize}
		\item<+-> 统一优化定权
			\begin{enumerate}[<+->]
			\item 分析二者定权的理论依据,联合定权解决差异
			\item 数值定量精化,结果尽量客观
			\end{enumerate}
		\item<+-> 针对CAP、CPS给出结果误差评价
			\begin{enumerate}[<+->]
			\item 误差方差
			\item 相关系数
			\end{enumerate}
	\end{itemize}
}
\section{实验检验}
\frame{
	\frametitle{secname}
	\begin{itemize}
		\item<+-> Install
		\item<+-> learn
		\item<+-> practise
	\end{itemize}
}
\section{案例应用}
\frame{
	\frametitle{secname}
	\begin{itemize}
		\item<+-> Install
		\item<+-> learn
		\item<+-> practise
	\end{itemize}
}
\section{总结和展望}
\frame{
	\frametitle{secname}
	\begin{itemize}
		\item<+-> Install
		\item<+-> learn
		\item<+-> practise
	\end{itemize}
}
\end{document}
